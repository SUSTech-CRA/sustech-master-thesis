% !TeX encoding = UTF-8
% !TeX program = xelatex
% !TeX spellcheck = en_US

\documentclass[degree=master,language=chinese,font=external,cjk-font=external]{sustechthesis}
  %%%%%%%%%%%%%%%%%%%%%%%%
  %   研究生学位论文模板
  %%%%%%%%%%%%%%%%%%%%%%%%

  % 学位 degree:
  %   master (默认) | doctor
  % 语言 language:
  %   chinese (默认)| english
  % 英文字体 font
  %   auto (默认,自动选择系统自带字体)| external (包内字体)| times | termes | 等
  %   Windows 和 macOS 系统上,无需设定。系统自带对应字体,可以删除该参数。
  %   Unix 系统推荐使用包内字体,而非TeX自带的克隆版字体,以达到和其他系统一致的字体效果。
  %   Windows 系统上可以删除该参数,使用系统内置字体。
  % 中文字体 cjk-font
  %   auto (默认,自动选择系统自带字体)| external (包内字体)| windows | mac | 等
  %   在 **非Windows** 的系统上推荐使用包内字体,而非系统字体。
  %   以达到和 Windows 系统一致的字体效果。
  %   Windows 系统自带对应字体,可以删除该参数。


% 论文基本配置,加载宏包等全局配置
% 在此文件中可以选择
% 1. 生成的PDF为无空白页的用于电子版提交的版本 或 插入空白页的以便双面打印的版本
% 2. 学位学科门类(理学、工学、医学)
% 3. 培养单位
% 4. 作者姓名、指导教师等
% 5. 修改gongshuo的值, 默认为false代表生成学术型研究生毕业设计模板, 改为true则将生成专业型研究生毕业设计模板
% !TeX root = ./sustechthesis-example.tex

% 论文基本信息配置

\thusetup{
  %******************************
  % 注意:
  %   1. 配置里面不要出现**空行**
  %   2. 不需要的配置信息可以删除
  %   3. 建议先阅读文档中所有关于选项的说明
  %******************************
  %
  % 输出格式
  %   选择打印版(print)或用于提交的电子版(electronic),前者会插入空白页以便直接双面打印
  %
  output = electronic,
  %
  % 标题
  %   可使用“\\”命令手动控制换行
  %   如果需要使用副标题,取消 subtitle 和 subtitle* 的注释即可。
  %
  title  = {南方科技大学学位论文 \LaTeX{} 模板 (Support English) 使用示例文档 v\version},
  title* = {An Introduction to \LaTeX{} Thesis Template of Southern University of Science and Technology v\version},
  % subtitle = {可选的副标题可选的副标题可选的副标题可选的副标题可选的副标题可选的副标题},
  % subtitle* = {optional subtitle optional subtitle optional subtitle optional subtitle optional subtitle optional subtitle},
  %
  % 学位
  %
  degree-domain = {工学}, % 【中文】学科门类:可选理学、工学、医学
  degree-domain* = {Engineering}, % 【英文】学位等级:可选Science, Engineering, Medicine
  gongsuo = false, % 是否为专业型学位。专业型学位则填 true ,学术型或其他为 false 。
  %
  % 培养单位
  %   填写所属院系的全名
  %   超长英文系名可以手动换行
  department = {计算机科学与工程系},
  department* = {School of System Design and \\Intelligent Manufacturing},
  %
  % 学科
  %   1. 学术型学位
  %      获得一级学科授权的学科填写一级学科名称,其他填写二级学科名称
  %   2. 工程硕士
  %      工程领域名称
  %
  discipline  = {计算机科学与技术},
  discipline* = {Computer Science and Technology},
  %
  % 姓名
  %   英文用全拼,姓在前,名在后,姓和名的首字母大写,其余小写
  %
  author  = {李子强},
  author* = {Li Ziqiang},
  %
  % 指导教师
  %   中文姓名和职称之间以英文逗号“,”分开,下同
  %
  supervisor  = {某某某(Alice Bob)助理教授},
  supervisor* = {Assistant Professor Alice Bob},
  %
  % 日期
  %   使用 ISO 格式;默认为当前时间
  %   date 为第一页全中文大写日期,defense-date 为第二、三页的答辩日期。
  %   需要按 {年-月-日} 格式填写,如不显示“日”,可以随意填一个日期,但是不能为空。
  %
  date = {2010-12-20},
  defense-date = {2020-12-20},
  %
  % 密级
  %   公开, 秘密, 机密, 绝密
  %
  statesecrets={公开},
  %
  % 国内图书分类号,国际图书分类号
  %
  natclassifiedindex={TM301.2},
  intclassifiedindex={62-5},
}

% 载入所需的宏包

% 可以使用 nomencl 生成符号和缩略语说明
% \usepackage{nomencl}
% \makenomenclature

% 表格加脚注
\usepackage{threeparttable}

% 表格中支持跨行
\usepackage{multirow}

% 量和单位
\usepackage{siunitx}

% 定理类环境宏包
\usepackage{amsthm}
% 也可以使用 ntheorem
% \usepackage[amsmath,thmmarks,hyperref]{ntheorem}

%%%%%% 参考文献编译方式二选一,不要同时开启。
%%%% 选择一
%% 参考文献使用 BibTeX + natbib 宏包
%% 顺序编码制
\usepackage[sort&compress]{gbt7714}
\bibliographystyle{gbt7714-numerical}
\usepackage{bibunits}

%%%% 选择二(不兼容本模板,请勿使用)
%% 参考文献使用 BibLaTeX 宏包
% \usepackage[backend=biber,style=gb7714-2015]{biblatex}
%% 声明 BibLaTeX 的数据库
% \addbibresource{ref/refs.bib}

% 定义所有的图片文件在 figures 子目录下
\graphicspath{{figures/}}

% 数学命令
\newcommand\dif{\mathop{}\!\mathrm{d}}  % 微分符号

% hyperref 宏包在最后调用
\usepackage{hyperref}
\usepackage{ragged2e}

% 固定宽度的表格。放在 hyperref 之前的话,tabularx 里的 footnote 显示不出来。
\usepackage{tabularx}

% 跨页表格,必须在 hyperref 之后使用否则会报错。
\usepackage{longtable}



\begin{document}

% 封面
\maketitle

% 学位论文公开评阅人和答辩委员会名单
% !TeX root = ../sustechthesis-example.tex

% 填写说明:
% 1、各类名单按实际人数逐行填写,可增加或删除行,不留空行。
% 2、“公开评阅人名单”仅填写公开评阅人信息,不填写隐名评阅人信息。
% 3、填写完毕后请及时删除提示红框。

\begin{committee}[name={学位论文公开评阅人和答辩委员会名单}]

  \vspace{18bp}

  \forcenewcolumntype{C}[1]{@{}>{\centering\arraybackslash}p{#1}}

  \section*{公开评阅人名单}

  \begin{center}
    \begin{tabular}{C{3cm}C{3cm}C{9cm}@{}}
      刘XX & 教授   & 南方科技大学                    \\
      陈XX & 副教授 & XXXX大学                    \\
      杨XX & 研究员 & 中国XXXX科学院XXXXXXX研究所 \\
    \end{tabular}
  \end{center}


  \section*{答辩委员会名单}

  \begin{center}
    \begin{tabular}{C{2.75cm}C{2.98cm}C{4.63cm}C{4.63cm}@{}}
      主席 & 赵XX                  & 教授                    & 南方科技大学       \\
      委员 & 刘XX                  & 教授                    & 南方科技大学       \\
          & \multirow{2}{*}{杨XX} & \multirow{2}{*}{研究员} & 中国XXXX科学院 \\
          &                       &                         & XXXXXXX研究所  \\
          & 黄XX                  & 教授                    & XXXX大学       \\
          & 周XX                  & 副教授                  & XXXX大学       \\
      秘书 & 吴XX                  & 助理研究员              & 南方科技大学       \\
    \end{tabular}
  \end{center}

\end{committee}



% 也可以导入 Word 版转的 PDF 文件
% \begin{committee}[file=figures/scan-committee.pdf]
% \end{committee}


% 南方科技大学学位论文原创性声明和使用授权说明
% 本模版不会对扫描版的页码进行处理,建议定稿后打印声明页再插入编译,以免页码出错。
% 或者,使用其他 pdf 拼接软件也可达到替换声明页面的目的。
\statementcopyright % 生成未签名的声明
% \statementcopyright[scan-statement.pdf] % 插入已签名的声明文件(扫描版)

\frontmatter
% !TeX root = ../sustechthesis-example.tex

% 中英文摘要和关键字

\begin{abstract}
  论文的摘要是对论文研究内容和成果的高度概括。
  摘要应对论文所研究的问题及其研究目的进行描述,对研究方法和过程进行简单介绍,对研究成果和所得结论进行概括。
  摘要应具有独立性和自明性,其内容应包含与论文全文同等量的主要信息。
  使读者即使不阅读全文,通过摘要就能了解论文的总体内容和主要成果。

  论文摘要的书写应力求精确、简明。
  切忌写成对论文书写内容进行提要的形式,尤其要避免“第 1 章……;第 2 章……;……”这种或类似的陈述方式。

  以\textbf{中文}为正文撰写的论文中,博士论文摘要约 800~1000 字,硕士论文摘要的字数一般为 500 字左右,且篇幅限制在一页内书写。
  以\textbf{英文}为正文撰写的论文中,中文摘要字数要求为800~1000字(不分博硕)。

  关键词是为了文献标引工作、用以表示全文主要内容信息的单词或术语。
  关键词不超过 5 个,每个关键词中间用分号分隔。

  % 关键词用“英文逗号”分隔,输出时会自动处理为正确的分隔符
  \thusetup{
    keywords = {关键词 1, 关键词 2, 关键词 3, 关键词 4, 长长长长长长长长长长长长长长长长长长长长长长长长长关键词 5},
  }
\end{abstract}

\begin{abstract*}
  An abstract of a dissertation is a summary and extraction of research work and contributions.
  Included in an abstract should be description of research topic and research objective, brief introduction to methodology and research process, and summarization of conclusion and contributions of the research.
  An abstract should be characterized by independence and clarity and carry identical information with the dissertation.
  It should be such that the general idea and major contributions of the dissertation are conveyed without reading the dissertation.

  An abstract should be concise and to the point.
  It is a misunderstanding to make an abstract an outline of the dissertation and words “the first chapter”, “the second chapter” and the like should be avoided in the abstract.

  For thesis written in \textbf{Chinese} as the main text, the abstract of doctoral thesis is about 800 to 1000 words; and the abstract of the master's thesis is generally about 500 words; the length is limited to one page.
  For thesis written in \textbf{English} as the main text, the word count of Chinese abstracts is required to be 800 to 1000 words (regardless doctoral thesis or master's thesis).

  The abstract of the doctoral thesis is about 800 to 1000 words; the abstract of the master's thesis is generally about 500 words; the length is limited to one page.

  Keywords are terms used in a dissertation for indexing, reflecting core information of the dissertation.
  An abstract may contain a maximum of 5 keywords, with semi-colons used in between to separate one another.

  % Use comma as seperator when inputting
  \thusetup{
    keywords* = {keyword 1, keyword 2, keyword 3, keyword 4, looooooooooooooooooooong keyword 5},
  }
\end{abstract*}


% 目录
\tableofcontents

% 插图和附表清单
% \listoffiguresandtables  % 插图和附表清单
% \listoffigures           % 插图清单
% \listoftables            % 附表清单

% 符号对照表(非强制性要求,如果论文中所用符号不多,可以略去)
% !TeX root = ../sustechthesis-example.tex

% denotation 环境带一个可选参数,用来指定符号列的宽度(默认为 2.5cm),下面改3cm为例。
% 如果论文中使用了大量的物理量符号、标志、缩略词、专门计量单位、自定义名词和术语等, 应编写“符号和缩略语说明”。
% 论文中主要符号应全部采用法定单位, 严格执行《量和单位》(GB3100~3102-93)的有关规定、单位名称的书写,可以采用国际通用符号,也可以用中文名称,但全文应统一,不得两种混用。
% 缩略语应列出中英文全称。符号和缩略语说明排序方法先按拉丁字母大写、小写排序, 再按希腊字母大写、小写排序, 如下表所示:
% ABCDEFGHIJKLMNOPQRSRUVWXYZ
% abcdefghijklmnopqrstuvwxyz
% Alpha
% Beta
% Gamma
% Delta
% Epsilon
% Zeta
% Eta
% Theta
% Iota
% Kappa
% Lambda
% Mu
% Nu
% Xi
% Omicron
% Pi
% Rho
% Sigma
% Tau
% Upsilon
% Phi
% Chi
% Psi
% Omega
% alpha
% beta
% gamma
% delta
% epsilon
% zeta
% eta
% theta
% iota
% kappa
% lambda
% mu
% nu
% xi
% omicron
% pi
% rho
% sigma
% tau
% upsilon
% phi
% chi
% psi
% omega

% 希腊字母详见 https://xilazimu.net/


\begin{denotation}[3cm]
  \item[As-PPT]聚苯基不对称三嗪
  \item[DFT]密度泛函理论 (Density Functional Theory)
  \item[DMAsPPT]聚苯基不对称三嗪双模型化合物(水解实验模型化合物)
  \item[$E_a$]化学反应的活化能 (Activation Energy)
  \item[HMAsPPT]聚苯基不对称三嗪模型化合物的质子化产物
  \item[HMPBI]聚苯并咪唑模型化合物的质子化产物
  \item[HMPI]聚酰亚胺模型化合物的质子化产物
  \item[HMPPQ]聚苯基喹噁啉模型化合物的质子化产物
  \item[HMPY]聚吡咙模型化合物的质子化产物
  \item[HMSPPT]聚苯基对称三嗪模型化合物的质子化产物
  \item[HPCE]高效毛细管电泳色谱 (High Performance Capillary lectrophoresis)
  \item[HPLC]高效液相色谱 (High Performance Liquid Chromatography)
  \item[IRC]内禀反应坐标 (Intrinsic Reaction Coordinates)
  \item[LC-MS]液相色谱-质谱联用 (Liquid chromatography-Mass Spectrum)
  \item[MAsPPT]聚苯基不对称三嗪单模型化合物,3,5,6-三苯基-1,2,4-三嗪
  \item[MPBI]聚苯并咪唑模型化合物,N-苯基苯并咪唑
  \item[MPI]聚酰亚胺模型化合物,N-苯基邻苯酰亚胺
  \item[MPPQ]聚苯基喹噁啉模型化合物,3,4-二苯基苯并二嗪
  \item[MPY]聚吡咙模型化合物
  \item[MSPPT]聚苯基对称三嗪模型化合物,2,4,6-三苯基-1,3,5-三嗪
  \item[ONIOM]分层算法 (Our own N-layered Integrated molecular Orbital and molecular Mechanics)
  \item[PBI]聚苯并咪唑
  \item[PDT]热分解温度
  \item[PES]势能面 (Potential Energy Surface)
  \item[PI]聚酰亚胺
  \item[PMDA-BDA]均苯四酸二酐与联苯四胺合成的聚吡咙薄膜
  \item[PPQ]聚苯基喹噁啉
  \item[PY]聚吡咙
  \item[S-PPT]聚苯基对称三嗪
  \item[SCF]自洽场 (Self-Consistent Field)
  \item[SCRF]自洽反应场 (Self-Consistent Reaction Field)
  \item[TIC]总离子浓度 (Total Ion Content)
  \item[TS]过渡态 (Transition State)
  \item[TST]过渡态理论 (Transition State Theory)
  \item[ZPE]零点振动能 (Zero Vibration Energy)
  \item[\textit[ab initio]]基于第一原理的量子化学计算方法,常称从头算法
  \item[$\Delta G^\neq$]活化自由能(Activation Free Energy)
  \item[$\kappa$]传输系数 (Transmission Coefficient)
  \item[$\nu_i$]虚频 (Imaginary Frequency)
\end{denotation}



% 也可以使用 nomencl 宏包,需要在导言区
% \usepackage{nomencl}
% \makenomenclature

% 在这里输出符号说明
% \printnomenclature[3cm]

% 在正文中的任意为都可以标题
% \nomenclature{As-PPT}{聚苯基不对称三嗪}
% \nomenclature{DFT}{密度泛函理论 (Density Functional Theory)}
% \nomenclature{DMAsPPT}{聚苯基不对称三嗪双模型化合物(水解实验模型化合物)}
% \nomenclature{$E_a$}{化学反应的活化能 (Activation Energy)}
% \nomenclature{HMAsPPT}{聚苯基不对称三嗪模型化合物的质子化产物}
% \nomenclature{HMPBI}{聚苯并咪唑模型化合物的质子化产物}
% \nomenclature{HMPI}{聚酰亚胺模型化合物的质子化产物}
% \nomenclature{HMPPQ}{聚苯基喹噁啉模型化合物的质子化产物}
% \nomenclature{HMPY}{聚吡咙模型化合物的质子化产物}
% \nomenclature{HMSPPT}{聚苯基对称三嗪模型化合物的质子化产物}
% \nomenclature{HPCE}{高效毛细管电泳色谱 (High Performance Capillary lectrophoresis)}
% \nomenclature{HPLC}{高效液相色谱 (High Performance Liquid Chromatography)}
% \nomenclature{IRC}{内禀反应坐标 (Intrinsic Reaction Coordinates)}
% \nomenclature{LC-MS}{液相色谱-质谱联用 (Liquid chromatography-Mass Spectrum)}
% \nomenclature{MAsPPT}{聚苯基不对称三嗪单模型化合物,3,5,6-三苯基-1,2,4-三嗪}
% \nomenclature{MPBI}{聚苯并咪唑模型化合物,N-苯基苯并咪唑}
% \nomenclature{MPI}{聚酰亚胺模型化合物,N-苯基邻苯酰亚胺}
% \nomenclature{MPPQ}{聚苯基喹噁啉模型化合物,3,4-二苯基苯并二嗪}
% \nomenclature{MPY}{聚吡咙模型化合物}
% \nomenclature{MSPPT}{聚苯基对称三嗪模型化合物,2,4,6-三苯基-1,3,5-三嗪}
% \nomenclature{ONIOM}{分层算法 (Our own N-layered Integrated molecular Orbital and molecular Mechanics)}
% \nomenclature{PBI}{聚苯并咪唑}
% \nomenclature{PDT}{热分解温度}
% \nomenclature{PES}{势能面 (Potential Energy Surface)}
% \nomenclature{PI}{聚酰亚胺}
% \nomenclature{PMDA-BDA}{均苯四酸二酐与联苯四胺合成的聚吡咙薄膜}
% \nomenclature{PPQ}{聚苯基喹噁啉}
% \nomenclature{PY}{聚吡咙}
% \nomenclature{S-PPT}{聚苯基对称三嗪}
% \nomenclature{SCF}{自洽场 (Self-Consistent Field)}
% \nomenclature{SCRF}{自洽反应场 (Self-Consistent Reaction Field)}
% \nomenclature{TIC}{总离子浓度 (Total Ion Content)}
% \nomenclature{TS}{过渡态 (Transition State)}
% \nomenclature{TST}{过渡态理论 (Transition State Theory)}
% \nomenclature{ZPE}{零点振动能 (Zero Vibration Energy)}
% \nomenclature{\textit{ab initio}}{基于第一原理的量子化学计算方法,常称从头算法}
% \nomenclature{$\Delta G^\neq$}{活化自由能(Activation Free Energy)}
% \nomenclature{$\kappa$}{传输系数 (Transmission Coefficient)}
% \nomenclature{$\nu_i$}{虚频 (Imaginary Frequency)}



% 正文部分
\mainmatter
\input{data/chap01}
% !TeX root = ../sustechthesis-example.tex

\chapter{图表示例}

\section{插图}

图片通常在 \env{figure} 环境中使用 \cs{includegraphics} 插入,如图~\ref{fig:example} 的源代码。
建议矢量图片使用 PDF 格式,比如数据可视化的绘图;
照片应使用 JPG 格式;
其他的栅格图应使用无损的 PNG 格式。
注意,LaTeX 不支持 TIFF 格式;EPS 格式已经过时。

\begin{figure}
  \centering
  \includegraphics[width=0.6\linewidth]{example-image-a.pdf}
  \caption{示例图片}
  \label{fig:example}
\end{figure}

\begin{figure}
  \centering
  \includegraphics[width=0.6\linewidth, angle=90]{example-image-a.pdf}
  \caption{示例图片旋转90度}
  \label{fig:exampleRotate90}
\end{figure}

若图或表中有附注,采用英文小写字母顺序编号,附注写在图或表的下方。
% LaTeX 传统上一般将附注的内容同图表的标题写在一起,形成很长的一段文字。

如果一个图由两个或两个以上分图组成时,各分图分别以(a)、(b)、(c)...... 作为图序,并须有分图题。
推荐使用 \pkg{subcaption} 宏包来处理, 比如图~\ref{fig:subfig-a} 和图~\ref{fig:subfig-b}。

\begin{figure}
  \centering
  \subcaptionbox{分图 A\label{fig:subfig-a}}
    {\includegraphics[width=0.45\linewidth]{example-image-a.pdf}}
  \subcaptionbox{分图 B\label{fig:subfig-b}}
    {\includegraphics[width=0.45\linewidth]{example-image-b.pdf}}
  \caption{多个分图的示例}
  \label{fig:multi-image}
\end{figure}



\section{表格}

表应具有自明性。为使表格简洁易读,尽可能采用三线表,如表~\ref{tab:three-line}。
三条线可以使用 \pkg{booktabs} 宏包提供的命令生成。

\begin{table}
  \centering
  \caption{三线表示例}
  \begin{tabular}{ll}
    \toprule
    文件名          & 描述                         \\
    \midrule
    thuthesis.dtx   & 模板的源文件,包括文档和注释 \\
    thuthesis.cls   & 模板文件                     \\
    thuthesis-*.bst & BibTeX 参考文献表样式文件    \\
    thuthesis-*.bbx & BibLaTeX 参考文献表样式文件  \\
    thuthesis-*.cbx & BibLaTeX 引用样式文件        \\
    \bottomrule
  \end{tabular}
  \label{tab:three-line}
\end{table}

表格如果有附注,尤其是需要在表格中进行标注时,可以使用 \pkg{threeparttable} 宏包。使用英文小写字母 a、b、c……顺序编号。

\begin{table}
  \centering
  \begin{threeparttable}[c]
    \caption{带附注以及调整列宽的的表格示例}
    \label{tab:three-part-table}
  \begin{tabular}{C{2cm} L{4cm} R{6cm}}
    \toprule
    2cm          & 4cm & 6cm                         \\
    \midrule
    左右居中的2cm宽度左右居中的2cm宽度\tnote{a}   & 左右居左的4cm宽度左右居左的4cm宽度 & 左右居右的6cm宽度左右居右的6cm宽度\\
    左右居中的2cm宽度左右居中的2cm宽度\tnote{b}   & 左右居左的4cm宽度左右居左的4cm宽度 & 左右居右的6cm宽度左右居右的6cm宽度\\
    \bottomrule
  \end{tabular}
    \begin{tablenotes}
      \item [a] A的注释
      \item [b] B的注释
    \end{tablenotes}
  \end{threeparttable}
\end{table}
如果需要调整表格列宽度, 可以改用命令 \verb|L|, \verb|R|, 或者 \verb|C|, 如 \verb|C{2cm}| 代表居中列宽2cm。

\begin{table}
  \centering
  \caption{合并单元格的三线表}
  \label{tab:merge-cell}
  \begin{tabular}{lcccc}
  \toprule
    Metaclass & \multicolumn{2}{c}{A-B} & \multicolumn{2}{c}{C-D} \\ \cmidrule(l){1-1} \cmidrule(lr){2-3} \cmidrule(r){4-5}
    Class & A & B & C & D \\ \midrule
    L1 & 1 & 2 & 3 & 4 \\
    L2 & 1 & 2 & 3 & 4 \\
  \bottomrule
  \end{tabular}
\end{table}

如有辅助线要求可以使用 \verb|\cmidrule| 命令。在连续使用时,可以使用一组圆括号括起来的参数 \verb|l|、\verb|r|或\verb|l<距离>|、\verb|r<距离>|表示间距的表格线可以在左右向内缩短一小段,表\ref{tab:merge-cell}展示了效果。

表格如果想要与页面等宽,可以使用 \pkg{tabularx} 宏包,如表格\ref{tab:textwith-example}所示。
模版定义了一些扩展命令,实现一些排版需求。\verb|X| 两端对齐, \verb|Y| 左对齐, \verb|Z| 右对齐,或者 \verb|A| 居中对齐。

\begin{table}
  \centering
  \caption{同页宽的表格实例}
  \label{tab:textwith-example}
  \begin{tabularx}{\textwidth}{YXAZ}
    \toprule
    Cell with text aligned to the left & 1 & 2 & 3\\ \midrule
    4 & Cell with justified text & 5 & 6\\ \midrule
    7 & 8 & Cell with centered text & 9\\ \midrule
    10 & 11 & 12 & Cell with text aligned to the right \\
    \bottomrule
  \end{tabularx}
\end{table}



如果您要排版的表格长度超过一页,那么推荐使用 \pkg{longtable} 或者 \pkg{supertabular}
宏包,模板对 \pkg{longtable} 进行了相应的设置,所以用起来可能简单一些。
表~\ref{tab:performance} 就是 \pkg{longtable} 的简单示例。

\begin{longtable}[c]{ccccclr}
  \caption{实验数据(超长表格示例)}\label{tab:performance}\\
  \toprule[1.5pt]
   测试程序 & \multicolumn{1}{c}{正常运行} & \multicolumn{1}{c}{同步} & \multicolumn{1}{c}{检查点} & \multicolumn{1}{c}{卷回恢复}
  & \multicolumn{1}{c}{进程迁移} & \multicolumn{1}{c}{检查点} \\
  & \multicolumn{1}{c}{时间 (s)}& \multicolumn{1}{c}{时间 (s)}&
  \multicolumn{1}{c}{时间 (s)}& \multicolumn{1}{c}{时间 (s)}& \multicolumn{1}{c}{
    时间 (s)}&  文件(KB)\\\midrule[1pt]
  \endfirsthead
    \multicolumn{7}{c}{\continuetable 实验数据(超长表格示例)}\\
  \toprule[1.5pt]
   测试程序 & \multicolumn{1}{c}{正常运行} & \multicolumn{1}{c}{同步} & \multicolumn{1}{c}{检查点} & \multicolumn{1}{c}{卷回恢复}
  & \multicolumn{1}{c}{进程迁移} & \multicolumn{1}{c}{检查点} \\
  & \multicolumn{1}{c}{时间 (s)}& \multicolumn{1}{c}{时间 (s)}&
  \multicolumn{1}{c}{时间 (s)}& \multicolumn{1}{c}{时间 (s)}& \multicolumn{1}{c}{
    时间 (s)}&  文件(KB)\\\midrule[1pt]
  \endhead
  \bottomrule[1.5pt]
  \multicolumn{7}{r}{续下页}
  \endfoot
  \endlastfoot
  CG.A.2 & 23.05 & 0.002 & 0.116 & 0.035 & 0.589 & 32491 \\
  CG.A.4 & 15.06 & 0.003 & 0.067 & 0.021 & 0.351 & 18211 \\
  CG.A.8 & 13.38 & 0.004 & 0.072 & 0.023 & 0.210 & 9890 \\
  CG.B.2 & 867.45 & 0.002 & 0.864 & 0.232 & 3.256 & 228562 \\
  CG.B.4 & 501.61 & 0.003 & 0.438 & 0.136 & 2.075 & 123862 \\
  CG.B.8 & 384.65 & 0.004 & 0.457 & 0.108 & 1.235 & 63777 \\
  MG.A.2 & 112.27 & 0.002 & 0.846 & 0.237 & 3.930 & 236473 \\
  MG.A.4 & 59.84 & 0.003 & 0.442 & 0.128 & 2.070 & 123875 \\
  MG.A.8 & 31.38 & 0.003 & 0.476 & 0.114 & 1.041 & 60627 \\
  MG.B.2 & 526.28 & 0.002 & 0.821 & 0.238 & 4.176 & 236635 \\
  MG.B.4 & 280.11 & 0.003 & 0.432 & 0.130 & 1.706 & 123793 \\
  MG.B.8 & 148.29 & 0.003 & 0.442 & 0.116 & 0.893 & 60600 \\
  LU.A.2 & 2116.54 & 0.002 & 0.110 & 0.030 & 0.532 & 28754 \\
  LU.A.4 & 1102.50 & 0.002 & 0.069 & 0.017 & 0.255 & 14915 \\
  LU.A.8 & 574.47 & 0.003 & 0.067 & 0.016 & 0.192 & 8655 \\
  LU.B.2 & 9712.87 & 0.002 & 0.357 & 0.104 & 1.734 & 101975 \\
  LU.B.4 & 4757.80 & 0.003 & 0.190 & 0.056 & 0.808 & 53522 \\
  LU.B.8 & 2444.05 & 0.004 & 0.222 & 0.057 & 0.548 & 30134 \\
  EP.A.2 & 123.81 & 0.002 & 0.010 & 0.003 & 0.074 & 1834 \\
  EP.A.4 & 61.92 & 0.003 & 0.011 & 0.004 & 0.073 & 1743 \\
  EP.A.8 & 31.06 & 0.004 & 0.017 & 0.005 & 0.073 & 1661 \\
  EP.B.2 & 495.49 & 0.001 & 0.009 & 0.003 & 0.196 & 2011 \\
  EP.B.4 & 247.69 & 0.002 & 0.012 & 0.004 & 0.122 & 1663 \\
  EP.B.8 & 126.74 & 0.003 & 0.017 & 0.005 & 0.083 & 1656 \\
  EP.A.2 & 123.81 & 0.002 & 0.010 & 0.003 & 0.074 & 1834 \\
  \bottomrule[1.5pt]
\end{longtable}


\section{源代码}

% 推荐使用 \pkg{listing} 环境嵌入 \pkg{minted} 环境高亮代码。\verb|linenos| 参数控制代码行号显示。\pkg{minted} 环境需要 Python 环境编译,并安装 Pygement 包,否则会编译失败。
% 引用效果如代码 \ref{lst:sample-code-minted}。

% 也可以
使用 \pkg{listings} 环境高亮代码。参数较为复杂,请自行搜索或查阅文档。引用效果如代码 \ref{lst:sample-code-listings}。示例使用 \pkg{minipage} 环境嵌套一层的原因是防止换页中被插入其他浮动体,结合实际情况,按需使用 \pkg{minipage},例如如需要跨页代码就无需使用 \pkg{minipage}。
% 但是,\textbf{不建议}混用 \pkg{listings} 环境和 \pkg{minted} 环境,会导致如上编号重复的错误,二选一即可。

% \begin{listing}[!ht]
% \caption{C++ 代码示例(使用 \pkg{minted} 高亮)}
% \label{lst:sample-code-minted}
% \begin{minted}[xleftmargin=20pt,linenos]{cpp}
% #include <cstdio>
% #include <cstdlib>
% #include <iostream>
% using namespace std;
% unsigned short i;
% int main() {
%   for (i = 0; i <= 5; i++) {
%     // whatever
%   }
%   return 0;
% }
% \end{minted}
% \end{listing}

\noindent% 取消 minipage 的缩进
\begin{minipage}{\linewidth}
\begin{lstlisting}[language=java,caption={Java 代码示例(使用 \pkg{listings} 高亮)},xleftmargin=20pt,label={lst:sample-code-listings}]
class HelloWorldApp {
    public static void main(String[] args) {
        System.out.println("Hello World!"); // Display the string.
        for (int i = 0; i < 100; ++i) {
            System.out.println(i);
        }
    }
}
\end{lstlisting}
\end{minipage}
 
\section{伪代码}
 

推荐使用 \pkg{algorithm2e} 宏包中的 \pkg{algorithm} 环境书写伪代码。\pkg{algorithm2e} 可选参数 \verb|linesnumbered| 控制代码行号显示。引用效果如算法 \ref{algo:sample-pseudocode}。

\begin{algorithm}
  \caption{Simulation-optimization heuristic}
  \label{algo:sample-pseudocode}
  \KwData{current period $t$, initial inventory $I_{t-1}$, initial capital $B_{t-1}$, demand samples}
  \KwResult{Optimal order quantity $Q^{\ast}_{t}$}
  $r\leftarrow t$\;
  $\Delta B^{\ast}\leftarrow -\infty$\;
  \While{$\Delta B\leq \Delta B^{\ast}$ \rm{and} $r\leq T$}{$Q\leftarrow\arg\max_{Q\geq 0}\Delta B^{Q}_{t,r}(I_{t-1},B_{t-1})$\;
  $\Delta B\leftarrow \Delta B^{Q}_{t,r}(I_{t-1},B_{t-1})/(r-t+1)$\;
  \If{$\Delta B\geq \Delta B^{\ast}$}{$Q^{\ast}\leftarrow Q$\;
  $\Delta B^{\ast}\leftarrow \Delta B$\;}
  $r\leftarrow r+1$\;}
\end{algorithm}


\begin{algorithm}
  \caption{SumExample}
  \label{algo:sample-pseudocode2}
  \KwResult{$s$}
  $s\leftarrow 0$ \Comment*[r]{这是默认多行注释}
  \Comment{这是默认独占一行的注释}
  \SetNoFillComment %
  \Comment{这是在取消独占一行后的注释}
  \SetFillComment
  \Comment{这是恢复独占一行的注释}
  % \LeftComment{左对齐}
  \ForEach{$i\in [1, 100]$}{
    \uIf{$i\% 3= 0$}{
      $s\leftarrow s+i$   \SingleComment*[r]{这是单行注释,一个没有end的if}
    }\uElseIf{$i\%3=1$}{
      \Break  \TriComment*{这是三角形的单行注释,一个没有end的else if}
    }\Else{
      \Continue \Comment*[r]{这是超长多行注释,关于伪代码的if-then-else详细查看https://texdoc.org/serve/algorithm2e/0的10.4}
    }
  }
  \Return{s} \;
\end{algorithm}

\newpage
\section{测试}

图题在图之下,段前空 6 磅,段后空 12 磅。图整体前后距离未定义,目前默认距离:段前空 12 磅,段后空 12 磅。

图前,图前,图前,图前,图前,图前,图前,图前,图前,图前,图前。

\begin{figure}[htbp]
  \centering
  \includegraphics[height=12bp,width=50bp]{example-image-a.pdf}\hspace*{6bp}
  \includegraphics[height=6bp,width=50bp]{example-image-a.pdf}
  \caption{图高度为12bp vs 6bp}
\end{figure}

图后,图后,图后,图后,图后,图后,图后,图后,图后,图后,图后。

表题在表之上,段前空 12 磅,段后空 6 磅。表整体前后距离未定义,目前默认距离:段前空 12 磅,段后空 12 磅。

表前,表前,表前,表前,表前,表前,表前,表前,表前,表前,表前。

\begin{table}[htbp]
  \centering
  \caption{简单表格}
    \begin{tabular}{cc}
    \toprule
    column1 & column2\\
    \midrule
    column1 & column2\\
    \bottomrule
    \end{tabular}
  \label{label}
\end{table}

表后,表后,表后,表后,表后,表后,表后,表后,表后,表后,表后。
\emph{图表前后是否有空行不影响图表与正文之间的距离}。

% !TeX root = ../sustechthesis-example.tex

\chapter{数学符号和公式}

\section{数学符号}

模板中使用 \pkg{unicode-math} 宏包来配置数学符号,

研究生《写作指南》要求量及其单位所使用的符号应符合国家标准《国际单位制及其应用》(GB 3100—1993)、《有关量、单位和符号的一般原则》(GB/T 3101—1993)的规定,但是与 \TeX{} 默认的美国数学学会(AMS)的符号习惯有所区别。

英文论文的数学符号使用 \TeX{} 默认的样式。论文以中文为主要撰写语言按照国标建议的配置数学字体格式:

\begin{enumerate}
  \item 大写希腊字母默认为斜体,如
    \begin{equation*}
      \Gamma \Delta \Theta \Lambda \Xi \Pi \Sigma \Upsilon \Phi \Psi \Omega.
    \end{equation*}
    注意有限增量符号 $\increment$ 固定使用正体,模板提供了 \cs{increment} 命令。
  \item 小于等于号和大于等于号使用倾斜的字形 $\le$、$\ge$。
  \item 积分号使用正体,比如 $\int$、$\oint$。
  \item 行间公式积分号的上下限位于积分号的上下两端,比如
    \begin{equation*}
      \int_a^b f(x) \dif x.
    \end{equation*}
    行内公式为了版面的美观,统一居右侧,如 $\int_a^b f(x) \dif x$ 。
  \item
    偏微分符号 $\partial$ 使用正体。
  \item
    省略号 \cs{dots} 按照中文的习惯固定居中,比如
    \begin{equation*}
      1, 2, \dots, n \quad 1 + 2 + \dots + n.
    \end{equation*}
  \item
    实部 $\Re$ 和虚部 $\Im$ 的字体使用罗马体。
\end{enumerate}

以上数学符号样式的差异可以在模板中统一设置。
另外国标还有一些与 AMS 不同的符号使用习惯,需要用户在写作时进行处理:
\begin{enumerate}
  \item 数学常数和特殊函数名用正体,如
    \begin{equation*}
      \uppi = 3.14\dots; \quad
      \symup{i}^2 = -1; \quad
      \symup{e} = \lim_{n \to \infty} \left( 1 + \frac{1}{n} \right)^n.
    \end{equation*}
  \item 微分号使用正体,比如 $\dif y / \dif x$。
  \item 向量、矩阵和张量用粗斜体(\cs{mathbfit}),如 $\mathbfit{x}$、$\mathbfit{\Sigma}$、$\mathbfit{T}$。
  \item 自然对数用 $\ln x$ 不用 $\log x$。
\end{enumerate}



关于数学符号更多的用法,参考
\href{http://mirrors.ctan.org/macros/latex/contrib/unicode-math/unicode-math.pdf}{\pkg{unicode-math}}
宏包的使用说明,
全部数学符号命的令参考
\href{http://mirrors.ctan.org/macros/latex/contrib/unicode-math/unimath-symbols.pdf}{\pkg{unimath-symbols}},也可以参考 Stack Overflow 上的答案 \href{https://tex.stackexchange.com/questions/58098/what-are-all-the-font-styles-i-can-use-in-math-mode}{What are all the font styles I can use in math mode?}。

关于量和单位推荐使用
\href{http://mirrors.ctan.org/macros/latex/contrib/siunitx/siunitx.pdf}{\pkg{siunitx}}
宏包,
可以方便地处理希腊字母以及数字与单位之间的空白,
比如:
\SI{6.4e6}{m},
\SI{9}{\micro\meter},
\si{kg.m.s^{-1}},
\SIrange{10}{20}{\degreeCelsius}。



\section{数学公式}

数学公式可以使用 \env{equation} 和 \env{equation*} 环境。
注意数学公式的引用应前后带括号,建议使用 \cs{eqref} 命令,比如式\eqref{eq:example}。
\begin{equation}
  \frac{1}{2 \uppi \symup{i}} \int_\gamma f = \sum_{k=1}^m n(\gamma; a_k) \mathscr{R}(f; a_k)
  \label{eq:example}
\end{equation}
注意公式编号的引用应含有圆括号,可以使用 \cs{eqref} 命令。

晶体衍射基础的著名公式──布拉格方程:
\begin{equation}
  2d\sin\theta=k\lambda, \quad k=1,2,3\ldots
\end{equation}

\noindent%
\begin{minipage}{\textwidth}
  \begin{tabularx}{\textwidth}{@{}l@{~}l@{~——~}X@{}}
    式中 & $d$ & 晶面间距(nm);\\
    & $\theta$ & 入射线与晶面的夹角(rad);\\
    & $\lambda$ & X射线波长(nm)。\\
    & $k$ & 公式中第一次出现的物理量代号应给予注释,注释的转行应与破折号“——”后第一个字对齐。
  \end{tabularx}
\end{minipage}

多行公式尽可能在“=”处对齐,推荐使用 \env{align} 环境。
\begin{align}
  a & = b + c + d + e \\
    & = f + g
\end{align}

此外需要注意:公式需紧挨段前文字,不可空行,不然会导致公式独立成段,如下\textcolor{red}{\textbf{错误}}效果。公式前文字公式前文字公式前文字公式前文字公式前文字。

\begin{equation}
  \frac{1}{2 \uppi \symup{i}} \int_\gamma f = \sum_{k=1}^m n(\gamma; a_k) \mathscr{R}(f; a_k)
\end{equation}

公式后文字公式后文字公式后文字公式后文字公式后文字公式后文字公式后文字公式后文字公式后文字公式后文字公式后文字。\textbf{正确}效果,如下:
% <- 如需代码上空行需要在前面加上百分号·%
\begin{equation}
  \frac{1}{2 \uppi \symup{i}} \int_\gamma f = \sum_{k=1}^m n(\gamma; a_k) \mathscr{R}(f; a_k)
\end{equation}
公式后文字公式后文字公式后文字公式后文字公式后文字公式后文字公式后。

\section{数学定理}

定理环境的格式可以使用 \pkg{amsthm} 或者 \pkg{ntheorem} 宏包配置。
用户在导言区载入这两者之一后,模板会自动配置 \env{thoerem}、\env{proof} 等环境。

\begin{theorem}[Lindeberg--Lévy 中心极限定理]
  设随机变量 $X_1, X_2, \dots, X_n$ 独立同分布, 且具有期望 $\mu$ 和有限的方差 $\sigma^2 \ne 0$,
  记 $\bar{X}_n = \frac{1}{n} \sum_{i+1}^n X_i$,则
  \begin{equation}
    \lim_{n \to \infty} P \left(\frac{\sqrt{n} \left( \bar{X}_n - \mu \right)}{\sigma} \le z \right) = \Phi(z),
  \end{equation}
  其中 $\Phi(z)$ 是标准正态分布的分布函数。
\end{theorem}
\begin{proof}
  Trivial.
\end{proof}

同时模板还提供了 \env{assumption}、\env{definition}、\env{proposition}、
\env{lemma}、\env{theorem}、\env{axiom}、\env{corollary}、\env{exercise}、
\env{example}、\env{remar}、\env{problem}、\env{conjecture} 这些相关的环境。

\section{数学字体}

按照《撰写规范》表达式字体可以采用 Times New Roman、Xits Math 或 Cambria Math(MS Word默认字体)。Cambria Math 缺少部分样式,例如:积分符号设定为 upright 也看起来没有变化。

TeX Gyre Termes Math 字体(Times New Roman的TeX克隆版)样例:
\makeatletter
\thu@load@math@font@times
\begin{equation}
  \frac{1}{2 \uppi \symup{i}} \int_\gamma f = \sum_{k=1}^m n(\gamma; a_k) \mathscr{R}(f; a_k)
\end{equation}
\makeatother

Cambria Math 字体样例:
\makeatletter
\thu@load@math@font@cambria
\begin{equation}
  \frac{1}{2 \uppi \symup{i}} \int_\gamma f = \sum_{k=1}^m n(\gamma; a_k) \mathscr{R}(f; a_k)
\end{equation}
\makeatother

Xits Math 字体样例:
\makeatletter
\thu@load@math@font@xits
\begin{equation}
  \frac{1}{2 \uppi \symup{i}} \int_\gamma f = \sum_{k=1}^m n(\gamma; a_k) \mathscr{R}(f; a_k)
\end{equation}
\thu@load@math@font@cambria
\makeatother

STIX Math 字体样例:
\makeatletter
\thu@load@math@font@stix
\begin{equation}
  \frac{1}{2 \uppi \symup{i}} \int_\gamma f = \sum_{k=1}^m n(\gamma; a_k) \mathscr{R}(f; a_k)
\end{equation}
\makeatother


% !TeX root = ../thuthesis-example.tex

\chapter{引用文献的标注}

模板支持 BibTeX 和 BibLaTeX 两种方式处理参考文献。
下文主要介绍 BibTeX 配合 \pkg{natbib} 宏包的主要使用方法。


\section{顺序编码制}

依学校样式规定,一般使用 \cs{cite},即序号置于方括号中,引文页码会放在括号外。统一处引用的连续序号会自动用短横线连接。

如多次引用同一文献,可能需要标注页码,例如:引用第二页\cite[2]{zhangkun1994},引用第五页\cite[5]{zhangkun1994}。

\thusetup{
  cite-style = super,
}
\begin{tabular}{l@{\quad$\Rightarrow$\quad}l}
  \verb|\cite{zhangkun1994}|               & \cite{zhangkun1994}   {\kaishu 不带页码的上标引用}            \\
  \verb|\cite[42]{zhangkun1994}|           & \cite[42]{zhangkun1994} {\kaishu 手动带页码的上标引用}          \\
  \verb|\cite{zhangkun1994,zhukezhen1973}| & \cite{zhangkun1994,zhukezhen1973}  {\kaishu 一次多篇文献的上标引用}  \\
\end{tabular}

注意,引文参考文献的每条都要在正文中标注
\cite{zhangkun1994,zhukezhen1973,dupont1974bone,zhengkaiqing1987,%
  jiangxizhou1980,jianduju1994,merkt1995rotational,mellinger1996laser,%
  bixon1996dynamics,mahui1995,carlson1981two,taylor1983scanning,%
  taylor1981study,shimizu1983laser,atkinson1982experimental,%
  kusch1975perturbations,guangxi1993,huosini1989guwu,wangfuzhi1865songlun,%
  zhaoyaodong1998xinshidai,biaozhunhua2002tushu,chubanzhuanye2004,%
  who1970factors,peebles2001probability,baishunong1998zhiwu,%
  weinstein1974pathogenic,hanjiren1985lun,dizhi1936dizhi,%
  tushuguan1957tushuguanxue,aaas1883science,fugang2000fengsha,%
  xiaoyu2001chubanye,oclc2000about,scitor2000project%
}。

引用测试:2个连续引用\cite{zhangkun1994,zhukezhen1973},2个间隔\cite{zhangkun1994,dupont1974bone},3个连续\cite{zhangkun1994,zhukezhen1973,dupont1974bone}。

\subsection{支持三级目录显示}

支持三级目录显示
% !TeX root = ../sustechthesis-example.tex

\chapter{English and $\text{lower-case}$ Example}

If your supervisor is a foreign resident, or if your supervisor or defense committee specifically allows writing in English, the thesis may be written in English as the primary language. Please check with your supervisor or department secretary to confirm if you can write in English.

\section{Reference guide}

Writing in English still requires the Chinese reference standard GB/T 7714-2015.



% 结论
\backmatter
% !TeX root = ../sustechthesis-example-bachelor.tex
% !Mode:: "TeX:UTF-8"

\chapter{结论}

本文介绍了南方科技大学本科毕业论文 \LaTeX\ 模板的使用方法。

主要结论如下:
\begin{enumerate}
  \item 本模板遵循学校的毕业论文撰写规范;
  \item 使用 \LaTeX\ 可以获得高质量的排版效果;
  \item 模板提供了丰富的中文接口,便于使用。
\end{enumerate}



% 参考文献
\bibliography{ref/refs}  % 参考文献使用 BibTeX 编译
% \printbibliography       % 参考文献使用 BibLaTeX 编译(兼容性不佳,不太推荐)

% 附录
\appendix
% !TeX root = ../sustechthesis-example.tex

\chapter{补充内容}

附录是与论文内容密切相关、但编入正文又影响整篇论文编排的条理和逻辑性的资料,例如某些重要的数据表格、计算程序、统计表等,是论文主体的补充内容,可根据需要设置。


\section{图表示例}

\subsection{图}

附录中的图片示例(图~\ref{fig:appendix-figure})。

\begin{figure}
  \centering
  \includegraphics[width=0.6\linewidth]{example-image-a.pdf}
  \caption{附录中的图片示例}
  \label{fig:appendix-figure}
\end{figure}


\subsection{表格}

附录中的表格示例(表~\ref{tab:appendix-table})。

\begin{table}
  \centering
  \caption{附录中的表格示例}
  \begin{tabular}{ll}
    \toprule
    文件名          & 描述                         \\
    \midrule
    sustechthesis.dtx   & 模板的源文件,包括文档和注释 \\
    sustechthesis.cls   & 模板文件                     \\
    thuthesis-*.bst & BibTeX 参考文献表样式文件    \\
    thuthesis-*.bbx & BibLaTeX 参考文献表样式文件  \\
    thuthesis-*.cbx & BibLaTeX 引用样式文件        \\
    \bottomrule
  \end{tabular}
  \label{tab:appendix-table}
\end{table}


\section{数学公式}

附录中的数学公式示例(公式\eqref{eq:appendix-equation})。
\begin{equation}
  \frac{1}{2 \uppi \symup{i}} \int_\gamma f = \sum_{k=1}^m n(\gamma; a_k) \mathscr{R}(f; a_k)
  \label{eq:appendix-equation}
\end{equation}


\section{源代码}

附录中的代码示例:
% 代码\ref{lst:appendix-sample-code-minted},
代码\ref{lst:appendix-sample-code-listings}。

% \begin{listing}[!ht]
% \caption{C++ 代码示例(使用 \pkg{minted} 高亮)}
% \label{lst:appendix-sample-code-minted}
% \begin{minted}[xleftmargin=20pt,linenos]{cpp}
% #include <cstdio>
% #include <cstdlib>
% #include <iostream>
% using namespace std;
% unsigned short i;
% int main() {
%   for (i = 0; i <= 5; i++) {
%     // whatever
%   }
%   return 0;
% }
% \end{minted}
% \end{listing}

\noindent% 取消 minipage 的缩进
\begin{minipage}{\linewidth}
\begin{lstlisting}[language=java,caption={Java 代码示例(使用 \pkg{listings} 高亮)},xleftmargin=20pt,label={lst:appendix-sample-code-listings}]
class HelloWorldApp {
    public static void main(String[] args) {
        System.out.println("Hello World!"); // Display the string.
        for (int i = 0; i < 100; ++i) {
            System.out.println(i);
        }
    }
}
\end{lstlisting}
\end{minipage}

\section{伪代码}

附录中的伪代码示例(算法\ref{algo:appendix-sample-pseudocode})。

\begin{algorithm}
  \caption{Simulation-optimization heuristic}
  \label{algo:appendix-sample-pseudocode}
  \KwData{current period $t$, initial inventory $I_{t-1}$, initial capital $B_{t-1}$, demand samples}
  \KwResult{Optimal order quantity $Q^{\ast}_{t}$}
  $r\leftarrow t$\;
  $\Delta B^{\ast}\leftarrow -\infty$\;
  \While{$\Delta B\leq \Delta B^{\ast}$ and $r\leq T$}{$Q\leftarrow\arg\max_{Q\geq 0}\Delta B^{Q}_{t,r}(I_{t-1},B_{t-1})$\;
  $\Delta B\leftarrow \Delta B^{Q}_{t,r}(I_{t-1},B_{t-1})/(r-t+1)$\;
  \If{$\Delta B\geq \Delta B^{\ast}$}{$Q^{\ast}\leftarrow Q$\;
  $\Delta B^{\ast}\leftarrow \Delta B$\;}
  $r\leftarrow r+1$\;}
\end{algorithm}


% 致谢
% !TeX root = ../sustechthesis-example.tex

\begin{acknowledgements}

衷心感谢导师×××教授对本人的精心指导。他的言传身教将使我终生受益。

感谢×××教授,以及实验室全体老师和同窗们的热情帮助和支持!

本课题承蒙××××基金资助,特此致谢。\\

\textcolor{red}{\textbf{以下内容为提示,仔细阅读后删除。}}

致谢应另起页,放置在参考文献、附录之后,标题和页眉均为“致谢”。语言要诚恳、恰当、简短。

致谢对象可以包括指导教师,在研究工作中提出建议和提供帮助的人,给予转载和引用权的资料、图片、文献、研究和调查的所有者,其他应感谢的组织和个人,资助研究工作的项目基金、奖学金基金、合同单位、资助或支持的企业、组织或个人,协助完成研究工作和提供便利条件的组织或个人。致谢字数以不超过一页纸为宜。

学位论文应由学生在导师(组)的指导下独立完成;\textcolor{red}{\textbf{若涉及团队工作,应注明属于团队成果,并明确个人独立完成的内容}},科学严谨,恪守规范。

\end{acknowledgements}


% 个人简历、在学期间完成的相关学术成果
% !TeX root = ../sustechthesis-example.tex

\begin{resume}

  \section*{个人简历} % 根据正文撰写语言选择

  ××××年××月××日出生于××××。

  ××××年××月考入××大学××院(系)××专业,××××年××月本科毕业并获得××学学士学位。

  ××××年××月——××××年××月,在××大学××院(系)××学科学习并攻读(获得)××学硕士学位。【注:博士生已获得硕士学位写“获得”,硕士生申请硕士学位应写“攻读”,本括号在使用时请删除】

  获奖情况:如获三好学生、优秀团干部、×奖学金等(不含科研学术获奖)。

  工作经历:……

  % \section*{Resume} % 根据正文撰写语言选择
  % FamilyName GivenName was born in 1997, in Shenzhen, Guangdong, China.

  % In September 2015, he/she was admitted to Southern University of Science and Technology (SUSTech). In June 2019, he/she obtained a bachelor's degree in engineering from the Department of Computer Science and Engineering, SUSTech.【注:此行填写已获得的本科学士学位】

  % In September 2019, he/she began his/her graduate study in the Department of Computer Science and Engineering, SUSTech, and got a master of engineering degree in Electronic Science and Technology, in July 2022.【注:未获得硕士学位的学生无需此行】

  % Since September 2022, he/she has started to pursue his/her master/doctor's degree of engineering in Electronic Science and Technology in the Department of Computer Science and Engineering, SUSTech.【注:此行填写正在攻读学位】

  % Awards: XXXX scholarship, SUSTech, 2019.

  % Work experience: XXXX Corp., Software engineer Intern (June 2021 - August 2021); XXXX Corp., Software engineer Intern (June 2021 - August 2021).

  \section*{在学期间完成的相关学术成果}
  % \section*{Academic Achievements during the Study for an Academic Degree}

  特别注意,下面的引用文献部分需要使用半角括号,例如[J],(已被xxxx录用)。(本行在使用时请删除)。

  \subsection*{学术论文}
  % \subsection*{Academic Articles}

  \begin{achievements}
    \item Pei S, Huang L L, Li G, et al. Magnetic Raman continuum in single-crystalline $\mathrm{H_3LiIr_2O_6}$[J]. Physical Review B, 2020, 101(20): 201101. (SCI收录, IDS号为LJ4UN, IF=3. 575, 对应学位论文2.2节和第5章.)
    \item Pei S, Tang J, Liu C, et al. Orbital-fluctuation freezing and magnetic-nonmagnetic phase transition in $\mathrm{α-TiBr_3}$[J]. Applied Physics Letters, 2020, 117(13): 133103. (SCI收录, IDS号为NY3GK, IF=3. 597, 对应学位论文2.2节和第3章.)
  \end{achievements}

  \subsection*{申请及已获得的专利(无专利时此项不必列出)}
  % \subsection*{Patents}

  \begin{achievements}
    \item 任天令, 杨轶, 朱一平, 等. 硅基铁电微声学传感器畴极化区域控制和电极连接的方法: 中国, CN1602118A[P]. 2005-03-30.
    \item Ren T L, Yang Y, Zhu Y P, et al. Piezoelectric micro acoustic sensor based on ferroelectric materials: USA, No.11/215, 102[P]. (美国发明专利申请号.)
  \end{achievements}

  \subsection*{参与的科研项目及获奖情况(无获奖时此项不必列出)}
  \begin{achievements}
    \item 姜锡洲,×××××研究,××省自然科学基金项目。课题编号:××××,长长长长长长长长长长长长长长长长长长长长长长长长长长长长长长长长长长长长长长长长长长长。
    \item ×××,×××××研究,××省自然科学基金项目。课题编号:××××。
    \item ×××,×××××研究,××省自然科学基金项目。课题编号:××××。
  \end{achievements}

\end{resume}


\end{document}
