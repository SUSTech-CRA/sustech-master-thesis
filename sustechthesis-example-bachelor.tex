% !TeX encoding = UTF-8
% !TeX program = xelatex
% !TeX spellcheck = zh_CN

\documentclass[degree=bachelor,language=chinese,font=external,cjk-font=external]{sustechthesis}
  %%%%%%%%%%%%%%%%%%%%%%%%
  %   本科生学位论文模板
  %%%%%%%%%%%%%%%%%%%%%%%%

  % 学位 degree:
  %   bachelor (本科生)
  % 语言 language:
  %   chinese (默认)| english
  % 英文字体 font
  %   auto (默认,自动选择系统自带字体)| external (包内字体)
  % 中文字体 cjk-font
  %   auto (默认,自动选择系统自带字体)| external (包内字体)

% 论文基本配置,加载宏包等全局配置
% !TeX root = ./sustechthesis-example-bachelor.tex

% 本科论文基本信息配置

\thusetup{
  %******************************
  % 注意:
  %   1. 配置里面不要出现**空行**
  %   2. 不需要的配置信息可以删除
  %   3. 建议先阅读文档中所有关于选项的说明
  %******************************
  %
  % 输出格式
  %   选择打印版(print)或用于提交的电子版(electronic),前者会插入空白页以便直接双面打印
  %
  output = electronic,
  %
  % 文档类型
  %   选择学位论文(thesis)【默认值】。
  %
  type = thesis,
  %
  % 标题
  %   可使用"\\"命令手动控制换行
  %   如果需要使用副标题,取消 subtitle 和 subtitle* 的注释即可。
  %
  title  = {南方科技大学本科毕业论文 \LaTeX{} 模板使用示例文档 v\version{}},
  title* = {An Introduction to \LaTeX{} Undergraduate Thesis Template of SUSTech v\version{}},
  % subtitle = {可选的副标题},
  % subtitle* = {optional subtitle},
  %
  % 培养单位
  %   填写所属院系的全名
  %
  department = {数学系},
  department* = {Department of Mathematics},
  %
  % 专业
  %
  discipline  = {信息与计算科学},
  discipline* = {Computational Mathematics},
  %
  % 姓名
  %   英文用全拼,姓在前,名在后,姓和名的首字母大写,其余小写
  %
  author-id  = {11711217},
  author  = {梁钰栋},
  author* = {Liang Yudong},
  %
  % 指导教师
  %   填写导师姓名,后衬导师职称"教授","副教授"等
  %
  supervisor  = {高德纳教授},
  supervisor* = {Prof. Donald E. Knuth},
  %
  % 日期
  %   使用 ISO 格式;默认为当前时间
  %
  date = {2025-06-01},
  %
  % 分类号(可选)
  %
  natclassifiedindex={TP311},
  intclassifiedindex={},
}

\thusetup{
  %
  % 数学字体
  %
  math-font  = cambria,
}

% 载入所需的宏包

% 表格加脚注
\usepackage{threeparttable}

% 表格中支持跨行
\usepackage{multirow}

% 量和单位
\usepackage{siunitx}

% 定理类环境宏包
\usepackage{amsthm}

% LaTeX logo
\usepackage{hologo}

% 参考文献
\usepackage[sort&compress]{natbib}
\bibliographystyle{sustechthesis-numeric}

% 定义所有的图片文件在 figures 子目录下
\graphicspath{{figures/}}

% 数学命令
\newcommand\dif{\mathop{}\!\mathrm{d}}

% hyperref 宏包
\usepackage{hyperref}

% 固定宽度的表格
\usepackage{tabularx}

% 跨页表格
\usepackage{longtable}

% 源代码高亮
\usepackage{listings}
\definecolor{javared}{rgb}{0.6,0,0}
\definecolor{javagreen}{rgb}{0.25,0.5,0.35}
\definecolor{javapurple}{rgb}{0.5,0,0.35}
\definecolor{javadocblue}{rgb}{0.25,0.35,0.75}

\lstset{language=Java,
  keywordstyle=\color{javapurple}\bfseries,
  stringstyle=\color{javared},
  commentstyle=\color{javagreen},
  morecomment=[s][\color{javadocblue}]{/**}{*/},
  numbers=left,
  numberstyle=\tiny\color{black},
  stepnumber=1,
  numbersep=10pt,
  tabsize=4,
  showspaces=false,
  showstringspaces=false
}

% 无意义文本(示例用)
\usepackage{zhlipsum,lipsum}

% 子图支持
\usepackage{subcaption}

% 三线表
\usepackage{booktabs}


\begin{document}

% 封面(包含中英文封面和诚信承诺书)
\maketitle

\frontmatter

% 摘要(使用新的摘要环境)
\begin{bachelorAbstract}{\LaTeX{};接口;本科论文;南方科技大学}
  如用英文写作,则英文摘要在前,中文摘要在后。

  笔者见到的毕业论文模板,大多是以文类的形式,少部分以宏包的形式,并且在模板中大多掺杂着各式各样的例子(除了维护频率高的模板),
  导致模板文件使用了大部分与形式格式不相关的内容,代码量巨大文档欠缺且不容易修改,出现问题需要查看宏包或者文类的源代码。
  于是,秉着仅提供实现最基本要求的理念,重构了之前所写的 \TeX\ 形式。
  由于第二年使用该模板,所以设计出的模板接口不能保证以后不发生重大变动,一切以文档为主。
  毕竟学校在发展初期,各类文件都在日渐完善,前几年时,学校标志及名称还发生变化,同时毕业论文的样式也发生了重大变化。
  但是可以保证的是,模板提供的接口均为中文形式\footnote{
    使用 \hologo{XeLaTeX} 特性,一方面增加辨识度,另一方面不拘泥于英文命名的规则。
    当然此举也有些许弊端,在此就不过多展开。
  },并且至少更新到 2021 年,也就是笔者毕业。
  模板这种东西不能保证一劳永逸,一方面学校的标准制度都在发生着改变,
  另一方面 \hologo{LaTeX} 的宏包也在发生着改变,早先流行的宏包可能几年后就被“淘汰”掉。
  因此,您的使用与反馈是我不断更新的动力,希望各位不吝赐教。
\end{bachelorAbstract}

\begin{bachelorAbstractEn}{LaTeX, Interface}
  For theses written in English, the English abstract should be placed before the Chinese abstract.

  \lipsum[1]
\end{bachelorAbstractEn}

% 目录
\tableofcontents

% 正文部分
\mainmatter
\input{bachelor-data/chap01}
% !TeX root = ../sustechthesis-example-bachelor.tex
% !Mode:: "TeX:UTF-8"

\chapter{文类接口}

文类的接口的命名均为汉字,意思为字面意思,
如有疑问,欢迎在 GitHub 提出 \href{https://github.com/SUSTech-CRA/sustech-master-thesis/issues}{Issues}。

\section{汉化字号接口}

本接口主要使用 \texttt{ctex} 宏包。

\verb|\初号|,\verb|\小初|,\verb|\一号|,\verb|\小一|,\verb|\二号|,\verb|\小二|,\verb|\三号|,\verb|\小三|,
\verb|\四号|,\verb|\小四|,\verb|\五号|,\verb|\小五|,\verb|\六号|,\verb|\小六|,\verb|\七号|,\verb|\八号|。

\section{汉化字体接口}

可能本机上部分字体不存在,导致部分字体无法使用。

\verb|\宋体|,\verb|\黑体|,\verb|\仿宋|,\verb|\楷书|,
\verb|\隶书|,\verb|\幼圆|,\verb|\雅黑|,\verb|\苹方|。

\section{字体效果接口}

建议在正文时使用 \verb|\textbf{}|,\verb|\textit{}| 调用\textbf{粗体}与\textit{斜体}。

It is recommended to use \verb|\textbf{}|,\verb|\textit{}| to call \textbf{Bold} and \textit{ItalicFont}.

\verb|\粗体|,\verb|\斜体|。

\section{格式相关接口}

\subsection{命令}

例子请参考前文,在写论文初期,可以注释掉标题页等不必要信息,以加快编译速度。

\verb|\设置信息|,\verb|\目录|,\verb|\下划线|,\verb|\中文标题页|,\verb|\英文标题页|,
\verb|\中文诚信承诺书|,\verb|\英文诚信承诺书|,\verb|\摘要标题|,\verb|\参考文献|,\verb|\附录|,\verb|\致谢|。

\subsection{环境}

摘要环境均需一个参数,为关键词:\verb|\begin{}{}...\end{}|。

\verb|中文摘要|,\verb|英文摘要|。

% !TeX root = ../sustechthesis-example-bachelor.tex
% !Mode:: "TeX:UTF-8"

\chapter{一些样例}

\section{表格}

表格与图片可以直接通过 \verb|\ref{<key>}| 来引用,例如表 \ref{tab:example}、图 \ref{fig:test-a}、图 \ref{fig:test-b-sub-b}。

\begin{table}[htb]
  \centering
  \caption{表格的标题应该放在上方}
  \label{tab:example}
  \begin{tabular}{lc}
    \toprule
    Example & Result \\
    \midrule
    Example1          & 0.25 \\
    Example2          & 0.36 \\
    \bottomrule
  \end{tabular}
\end{table}

\begin{table}[htb]
  \centering
  \caption{带表注的表格的标题}
  \label{tab:example2}
  \begin{threeparttable}
    \setlength{\tabcolsep}{0.6cm}{
      \begin{tabular}{lc}
        \toprule
        Example & Result \\
        \midrule
        Example1          & 0.25\tnote{1} \\
        Example2          & 0.36 \\
        \bottomrule
      \end{tabular}
    }
    \begin{tablenotes}
    \item[1] 数据来源:南方科技大学 \LaTeX 模版
    \end{tablenotes}
  \end{threeparttable}
\end{table}

\begin{proof}
  这是一个证明符号的示例。
\end{proof}

\section{参考文献}

参考文献一般使用 \verb|\cite{<key>}| 命令来引用。

\section{图片}

\begin{figure}[htb]
  \centering
  \includegraphics[width=.5\textwidth]{example-image-a}
  \caption{自带测试图片---Test image}\label{fig:test-a}
\end{figure}

\begin{figure}[htb]
  \centering
  \begin{subfigure}[t]{.45\linewidth}
    \centering
    \includegraphics[width=1\textwidth]{example-image-a}
    \caption{子图-自带测试图片---Test image}\label{fig:test-b-sub-a}
  \end{subfigure}
  \begin{subfigure}[t]{.45\linewidth}
    \centering
    \includegraphics[width=1\textwidth]{example-image-a}
    \caption{子图-自带测试图片---Test image}\label{fig:test-b-sub-b}
  \end{subfigure}
  \caption{自带测试图片---Test image}\label{fig:test-b}
\end{figure}

% !TeX root = ../sustechthesis-example-bachelor.tex
% !Mode:: "TeX:UTF-8"

\chapter{\LaTeX\ 入门}

请参考 \href{https://tex.readthedocs.io/zh_CN/latest/}{在线文档},包括学习资源及学习路径。
欢迎在 GitHub 上提出 \href{https://github.com/SUSTech-CRA/sustech-master-thesis/issues}{Issues}。


% 结论
\backmatter
% !TeX root = ../sustechthesis-example-bachelor.tex
% !Mode:: "TeX:UTF-8"

\chapter{结论}

本文介绍了南方科技大学本科毕业论文 \LaTeX\ 模板的使用方法。

主要结论如下:
\begin{enumerate}
  \item 本模板遵循学校的毕业论文撰写规范;
  \item 使用 \LaTeX\ 可以获得高质量的排版效果;
  \item 模板提供了丰富的中文接口,便于使用。
\end{enumerate}


% 参考文献(使用 biblatex)
\printbibliography

% 附录
\appendix
% !TeX root = ../sustechthesis-example-bachelor.tex
% !Mode:: "TeX:UTF-8"

\section{附录示例}

这是附录的示例内容。

\section{代码示例}

可以在此处放置代码或其他补充材料。

\begin{verbatim}
def hello_world():
    print("Hello, World!")
\end{verbatim}


% 致谢
% !TeX root = ../sustechthesis-example.tex

\begin{acknowledgements}

衷心感谢导师×××教授对本人的精心指导。他的言传身教将使我终生受益。

感谢×××教授,以及实验室全体老师和同窗们的热情帮助和支持!

本课题承蒙××××基金资助,特此致谢。\\

\textcolor{red}{\textbf{以下内容为提示,仔细阅读后删除。}}

致谢应另起页,放置在参考文献、附录之后,标题和页眉均为“致谢”。语言要诚恳、恰当、简短。

致谢对象可以包括指导教师,在研究工作中提出建议和提供帮助的人,给予转载和引用权的资料、图片、文献、研究和调查的所有者,其他应感谢的组织和个人,资助研究工作的项目基金、奖学金基金、合同单位、资助或支持的企业、组织或个人,协助完成研究工作和提供便利条件的组织或个人。致谢字数以不超过一页纸为宜。

学位论文应由学生在导师(组)的指导下独立完成;\textcolor{red}{\textbf{若涉及团队工作,应注明属于团队成果,并明确个人独立完成的内容}},科学严谨,恪守规范。

\end{acknowledgements}


\end{document}
