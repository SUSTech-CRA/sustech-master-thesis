% !TeX root = ../sustechthesis-example-bachelor.tex
% !Mode:: "TeX:UTF-8"

\chapter{文类接口}

文类的接口的命名均为汉字,意思为字面意思,
如有疑问,欢迎在 GitHub 提出 \href{https://github.com/SUSTech-CRA/sustech-master-thesis/issues}{Issues}。

\section{汉化字号接口}

本接口主要使用 \texttt{ctex} 宏包。

\verb|\初号|,\verb|\小初|,\verb|\一号|,\verb|\小一|,\verb|\二号|,\verb|\小二|,\verb|\三号|,\verb|\小三|,
\verb|\四号|,\verb|\小四|,\verb|\五号|,\verb|\小五|,\verb|\六号|,\verb|\小六|,\verb|\七号|,\verb|\八号|。

\section{汉化字体接口}

可能本机上部分字体不存在,导致部分字体无法使用。

\verb|\宋体|,\verb|\黑体|,\verb|\仿宋|,\verb|\楷书|,
\verb|\隶书|,\verb|\幼圆|,\verb|\雅黑|,\verb|\苹方|。

\section{字体效果接口}

建议在正文时使用 \verb|\textbf{}|,\verb|\textit{}| 调用\textbf{粗体}与\textit{斜体}。

It is recommended to use \verb|\textbf{}|,\verb|\textit{}| to call \textbf{Bold} and \textit{ItalicFont}.

\verb|\粗体|,\verb|\斜体|。

\section{格式相关接口}

\subsection{命令}

例子请参考前文,在写论文初期,可以注释掉标题页等不必要信息,以加快编译速度。

\verb|\设置信息|,\verb|\目录|,\verb|\下划线|,\verb|\中文标题页|,\verb|\英文标题页|,
\verb|\中文诚信承诺书|,\verb|\英文诚信承诺书|,\verb|\摘要标题|,\verb|\参考文献|,\verb|\附录|,\verb|\致谢|。

\subsection{环境}

摘要环境均需一个参数,为关键词:\verb|\begin{}{}...\end{}|。

\verb|中文摘要|,\verb|英文摘要|。
