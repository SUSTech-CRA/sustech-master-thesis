% !TeX root = ./sustechthesis-example-bachelor.tex

% 本科论文基本信息配置

\thusetup{
  %******************************
  % 注意:
  %   1. 配置里面不要出现**空行**
  %   2. 不需要的配置信息可以删除
  %   3. 建议先阅读文档中所有关于选项的说明
  %******************************
  %
  % 输出格式
  %   选择打印版(print)或用于提交的电子版(electronic),前者会插入空白页以便直接双面打印
  %
  output = electronic,
  %
  % 文档类型
  %   选择学位论文(thesis)【默认值】。
  %
  type = thesis,
  %
  % 标题
  %   可使用"\\"命令手动控制换行
  %   如果需要使用副标题,取消 subtitle 和 subtitle* 的注释即可。
  %
  title  = {南方科技大学本科毕业论文 \LaTeX{} 模板使用示例文档 v\version{}},
  title* = {An Introduction to \LaTeX{} Undergraduate Thesis Template of SUSTech v\version{}},
  % subtitle = {可选的副标题},
  % subtitle* = {optional subtitle},
  %
  % 培养单位
  %   填写所属院系的全名
  %
  department = {数学系},
  department* = {Department of Mathematics},
  %
  % 专业
  %
  discipline  = {信息与计算科学},
  discipline* = {Computational Mathematics},
  %
  % 姓名
  %   英文用全拼,姓在前,名在后,姓和名的首字母大写,其余小写
  %
  author-id  = {11711217},
  author  = {梁钰栋},
  author* = {Liang Yudong},
  %
  % 指导教师
  %   填写导师姓名,后衬导师职称"教授","副教授"等
  %
  supervisor  = {高德纳教授},
  supervisor* = {Prof. Donald E. Knuth},
  %
  % 日期
  %   使用 ISO 格式;默认为当前时间
  %
  date = {2025-06-01},
  %
  % 分类号(可选)
  %
  natclassifiedindex={TP311},
  intclassifiedindex={},
}

\thusetup{
  %
  % 数学字体
  %
  math-font  = cambria,
}

% 载入所需的宏包

% 表格加脚注
\usepackage{threeparttable}

% 表格中支持跨行
\usepackage{multirow}

% 量和单位
\usepackage{siunitx}

% 定理类环境宏包
\usepackage{amsthm}

% LaTeX logo
\usepackage{hologo}

% 参考文献
\usepackage[sort&compress]{natbib}
\bibliographystyle{sustechthesis-numeric}

% 定义所有的图片文件在 figures 子目录下
\graphicspath{{figures/}}

% 数学命令
\newcommand\dif{\mathop{}\!\mathrm{d}}

% hyperref 宏包
\usepackage{hyperref}

% 固定宽度的表格
\usepackage{tabularx}

% 跨页表格
\usepackage{longtable}

% 源代码高亮
\usepackage{listings}
\definecolor{javared}{rgb}{0.6,0,0}
\definecolor{javagreen}{rgb}{0.25,0.5,0.35}
\definecolor{javapurple}{rgb}{0.5,0,0.35}
\definecolor{javadocblue}{rgb}{0.25,0.35,0.75}

\lstset{language=Java,
  keywordstyle=\color{javapurple}\bfseries,
  stringstyle=\color{javared},
  commentstyle=\color{javagreen},
  morecomment=[s][\color{javadocblue}]{/**}{*/},
  numbers=left,
  numberstyle=\tiny\color{black},
  stepnumber=1,
  numbersep=10pt,
  tabsize=4,
  showspaces=false,
  showstringspaces=false
}

% 无意义文本(示例用)
\usepackage{zhlipsum,lipsum}

% 子图支持
\usepackage{subcaption}

% 三线表
\usepackage{booktabs}
