% !TeX root = ../sustechthesis-example.tex

\chapter{补充内容}

附录是与论文内容密切相关、但编入正文又影响整篇论文编排的条理和逻辑性的资料,例如某些重要的数据表格、计算程序、统计表等,是论文主体的补充内容,可根据需要设置。


\section{图表示例}

\subsection{图}

附录中的图片示例(图~\ref{fig:appendix-figure})。

\begin{figure}
  \centering
  \includegraphics[width=0.6\linewidth]{example-image-a.pdf}
  \caption{附录中的图片示例}
  \label{fig:appendix-figure}
\end{figure}


\subsection{表格}

附录中的表格示例(表~\ref{tab:appendix-table})。

\begin{table}
  \centering
  \caption{附录中的表格示例}
  \begin{tabular}{ll}
    \toprule
    文件名          & 描述                         \\
    \midrule
    sustechthesis.dtx   & 模板的源文件,包括文档和注释 \\
    sustechthesis.cls   & 模板文件                     \\
    thuthesis-*.bst & BibTeX 参考文献表样式文件    \\
    thuthesis-*.bbx & BibLaTeX 参考文献表样式文件  \\
    thuthesis-*.cbx & BibLaTeX 引用样式文件        \\
    \bottomrule
  \end{tabular}
  \label{tab:appendix-table}
\end{table}


\section{数学公式}

附录中的数学公式示例(公式\eqref{eq:appendix-equation})。
\begin{equation}
  \frac{1}{2 \uppi \symup{i}} \int_\gamma f = \sum_{k=1}^m n(\gamma; a_k) \mathscr{R}(f; a_k)
  \label{eq:appendix-equation}
\end{equation}


\section{源代码}

附录中的代码示例:
% 代码\ref{lst:appendix-sample-code-minted},
代码\ref{lst:appendix-sample-code-listings}。

% \begin{listing}[!ht]
% \caption{C++ 代码示例(使用 \pkg{minted} 高亮)}
% \label{lst:appendix-sample-code-minted}
% \begin{minted}[xleftmargin=20pt,linenos]{cpp}
% #include <cstdio>
% #include <cstdlib>
% #include <iostream>
% using namespace std;
% unsigned short i;
% int main() {
%   for (i = 0; i <= 5; i++) {
%     // whatever
%   }
%   return 0;
% }
% \end{minted}
% \end{listing}

\noindent% 取消 minipage 的缩进
\begin{minipage}{\linewidth}
\begin{lstlisting}[language=java,caption={Java 代码示例(使用 \pkg{listings} 高亮)},xleftmargin=20pt,label={lst:appendix-sample-code-listings}]
class HelloWorldApp {
    public static void main(String[] args) {
        System.out.println("Hello World!"); // Display the string.
        for (int i = 0; i < 100; ++i) {
            System.out.println(i);
        }
    }
}
\end{lstlisting}
\end{minipage}

\section{伪代码}

附录中的伪代码示例(算法\ref{algo:appendix-sample-pseudocode})。

\begin{algorithm}
  \caption{Simulation-optimization heuristic}
  \label{algo:appendix-sample-pseudocode}
  \KwData{current period $t$, initial inventory $I_{t-1}$, initial capital $B_{t-1}$, demand samples}
  \KwResult{Optimal order quantity $Q^{\ast}_{t}$}
  $r\leftarrow t$\;
  $\Delta B^{\ast}\leftarrow -\infty$\;
  \While{$\Delta B\leq \Delta B^{\ast}$ and $r\leq T$}{$Q\leftarrow\arg\max_{Q\geq 0}\Delta B^{Q}_{t,r}(I_{t-1},B_{t-1})$\;
  $\Delta B\leftarrow \Delta B^{Q}_{t,r}(I_{t-1},B_{t-1})/(r-t+1)$\;
  \If{$\Delta B\geq \Delta B^{\ast}$}{$Q^{\ast}\leftarrow Q$\;
  $\Delta B^{\ast}\leftarrow \Delta B$\;}
  $r\leftarrow r+1$\;}
\end{algorithm}
